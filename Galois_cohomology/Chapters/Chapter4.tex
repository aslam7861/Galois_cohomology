\chapter{Global Fundamental Class} % Main chapter title

\label{Chapter1} % For referencing the chapter elsewhere, use \ref{Chapter1}

\lhead{Chapter 4. \emph{Global Fundamental Class}} % This is for the header on each page - perhaps a shortened title

%----------------------------------------------------------------------------------------
\section{Global fundamental class}
{\color{blue}In this we have been through his thesis and found that few of his assumptions are not necessary to include in the algorithm. In this the expensive part is to check the criterion of S1 - S4 conditions. It is all to known that the computation of class group and computation of S-unit group are expensive. We can see how the time increase while computing the global fundamental class for higher degree field extensions. place it after algorithm}
In the global field $K$ the id\`{e}le class group $C_K$ plays the role what the multiplicative group of fields played in the local class field theory. Let $G=\Gal(\overline{K}/K)$ be the absolute Galois group where $\overline{K}$ is the separable closure of $K$. Let $\mathfrak{P}(K)$ be the set of all places of $K$ including archimedians. For a place $\mathfrak{p} \in \mathfrak{P}(K)$, $K_{\mathfrak{p}}$ is the completion of $K$ at $\mathfrak{p}$. The id\`ele group $I_K$ of $K$ is defined as $ I_K = \prod_{\mathfrak{p}}K_{\mathfrak{p}}^{\times}$ and the id\`ele class group $C_K$ of $K$ is defined as \[ C_K = I_K/K^{\times}. \]
Suppose $L/K$ be a finite Galois extension of number fields, $G= \Gal(L/K)$ and consider the exact sequence
\[ 0 \rightarrow L^{\times} \rightarrow I_{L} \rightarrow C_L .\]
Then we obtain a cohomology sequence
\[ 0 \rightarrow H^{0}(G,L^{\times}) \rightarrow H^{0}(G, I_{L}) \rightarrow H^{0}(G,C_L) \rightarrow H^{1}(G,L^{\times}).\]
This becomes
\[ 0 \rightarrow {L^{\times}}^G \rightarrow {I_{L}^{\times}}^G \rightarrow {C_{L}^{\times}}^G \rightarrow 0,   \]
since $H^{1}(G,L^{\times})=0$ and we get $C_{L}^{G}= {I_{L}^{\times}}^G/ {L^{\times}}^G =I_K/ K^{\times}=C_K$.

Instead of working over id\`ele class group $C_L$ we work over $S$-id\`ele class group $C_{L,S}$ which we describe below:

\textbf{Main theorem on the abelian extensions(TAKAGI-ARTIN):172 Cassel}
\begin{theorem}
\begin{enumerate}
\item Every abelian extension $L/K$ satisfies the reciprocity law (i.e. there is an Artin map $\psi_{L/K}$).
\item The Artin map $\psi_{L/K}$ is surjective with kernel $K^{*}N_{L/K}(J_{L})$ and hence induces an isomorphism of $C_{K}/N_{L/K}(C_L)$ onto $G(L/K)$.
\item If $M\supset L\supset K $ are the abelian extensions, then the diagram

\[\begin{tikzcd}
C_K/N_{M/K}C_M\arrow{d}{j}\arrow{r}{\psi_{M/K}} & G(M/K)\arrow{d}{\theta}\\
C_K/N_{M/K}C_M\arrow{r}{\psi_{L/K}} &
G(L/K)
\end{tikzcd}\]
commutes (wher $\theta$ is the usual map and $j$ is the natural surjective map which exists because $N_{M/K}C_M \subset N_{L/K}C_L$).
\item (\textbf{Existence Theorem.}) For every open subgroup $N$ of finite index in $C_K$ there exists a unique abelian extension $L/K$ (in a fixed algebraic closure of $K$) such that $N_{L/K} C_L = N$.


 \end{enumerate}
\end{theorem}


\textbf{Cohomology of Idele Classes:}
We have the exact sequences 
\[  0 \rightarrow L^{*} \rightarrow J_{L} \rightarrow C_L \rightarrow 0.\]
The action of $G$ on $C_L$ is that induced by its action on $J_L$.
\begin{proposition}
$C_K\simeq C_{L}^{G}.$
\end{proposition}
\begin{proof}
   The above exact sequence gives rise to cohomology sequence
   \[ 0\rightarrow H^{0}(G,L^{*})\rightarrow H^{0}(G,J_L)\rightarrow H^{0}(G,C_L)\rightarrow  H^{1}(G,L^{*}),\]                        
   
that is \[   0\rightarrow K^{*}\rightarrow J_K \rightarrow C_{L}^{G} \rightarrow 0.\]
\end{proof}
Let $K$ be a number field and $S$ a nonempty set of primes  in $K$  containing the set $S_{\infty}$ of infinite primes in $K$. The ring of $S$-integers of $K$ is defined as
\[ \mathcal{O}_{K,S}=\cap_{p\notin S}\mathcal{O}_p = \{a \in K \mid v_{p}(a)\geq 0 \mbox{    } \forall p \notin S\}\]
where $\mathcal{O}_p$ is the ring of integers of $K_{p}$.
Its group of units $\mathcal{O}_{K,S}^{\times}$  and its ideal class group $Cl_{S}(K)$ (it is the quotient of the usual ideal class group $Cl_{K}$ of $K$ by the subgroup  generated by the classes of all prime ideals in $S$) play a particularly important role. $Cl_{S}(K)$ is finite and is called the $S$-ideal class group.\\
Let $\overline{k}$ be the separable closure of the number field $k$. Assume that $k_{S}$ be the maximal subextension of $k$ in $\overline{k}$ which is unramified outside $S$. The ring of $k_{S}$ is denoted by 
\[  \mathcal{O}_{S} = \cup_{K/k}^{}\mathcal{O}_{K,S} ,\]  where the union is taken over all finite extensions $K$ of $k$ in $k_{S}$.\\
Now we know the $S$-id\`ele group \[ I_{K,S}:= \prod_{p \in S}K_{p}^{\times} .\]
It contains the group of $S$-units $\mathcal{O}_{K,S}^{\times}$ as a discrete subgroup and we set \[ C_{K,S}= I_{K,S}/\mathcal{O}_{K,S}^{\times}\].
In spite of the analogy with the formation of id\`ele class group $C_{K}=I_{K}/K^{\times}$, it is not this group which takes the role $C_{K}$ in the $S$-theory. The reason is that if we take $K/k$ as Galois with $G=\Gal(K/k)$ then $C_{k,S}$ is not always the fixed module $C_{K,S}^{G}$.\\
Consider the group \[ C_{S}(K)= I_{K}/{K^{\times}U_{K,S}}  \] instead of $C_{K,S}$, where $U_{K,S}$ is the compact subgroup
\[U_{K,S} =\prod_{p \in S}\{1\}  \times \prod_{p \notin S}U_{p} \]
of the full id\`ele group $I_{K}$ where $U_p$ is the group of units of $\mathcal{O}_p$. Since $K^{\times} \cap U_{K,S}= 1$ , we may regard $U_{K,S}$  as a subgroup of $C_{K}= I_{K}/K^{\times}$ and we may also write $C_{S}(K)=C_{K}/U_{K,S}  $. This group is called the "$S$-id\`ele class group".
 If $K/k$  is Galois then $U_{K,S}$ is cohomologically trivial since 
 \[ H^{i}(G,U_{K,S})= \sum_{p\notin S}H^{i}(G_p, U_{p})  \]
where cohomologies on right side are trivial because of the unramified extension. ($U_{p}^{i}/U_{p}^{i+1}$, for $i\geq 1$, is isomorphic $F_{K_p}$, residue class field of $K_p$ and since it is cohomologically trivial and $U_{p}^{1}= \lim\limits_{\leftarrow} U_{p}^{1}/U_{p}^{i}$ and so trivial. Also exact sec of $0 \rightarrow U_{p}^{1}\rightarrow U_{K_{p}} \rightarrow F_{K_{p}}^{\times} \rightarrow 0$ and trivial comes due to long exact seq). 


If $K/k$ is Galois, then we have the exact sequence of $G(K/k)$-modules
\[ 0\rightarrow U_{K,S} \rightarrow C_{K} \xrightarrow{\pi} C_{S}(K) \rightarrow 0.\]
\begin{theorem}
 For every finite Galois extension $K/k$ and every $i \in \mathbb{Z}$,
 \[ H^{i}(G(K/k),C_{S}(K)) \cong H^{i}(G(K/k),C_{K}).\]
\end{theorem}
\begin{proposition}(8.3.5)(Neukirch454)
$C_{K,S}$ is an open subgroup of $C_{S}(K)$ and there is an exact sequence 
\[\begin{tikzcd}[scale=1em]
\ 0 \arrow{r} & C_{K,S}\arrow{r} & C_{S}(K)\arrow{r}{\pi} & Cl_{S}(K)\arrow{r} & 0. 
\end{tikzcd}\]
%\[ 0 \rightarrow C_{K,S}\rightarrow C_{S}(K)\rightarrow^{\pi} Cl_{S}(K)\rightarrow 0 .\]
In particular, $C_{K,S}= C_{S}(K)$ if $S$ omits only finitely many primes.
\end{proposition}
\textbf{Note:} If $S$ omits only finitely many prime ideals, then $O_{K,S}$ is Dedekind ring with finitely many prime ideals and is hence a PID (because of page-454, Neukirch). Therefore in this case $Cl_{S}(K)=0$ and $C_{K,S}\simeq C_{S}(K)$.
We choose $S$ sufficiently large so that $Cl_{S}(K)=0$ so that we can work with $C_{L,S}$ instead of $C_{S}(K)$. Because of the finiteness of class group $Cl(K)$ of $K$ and every ideal class of $Cl(K)$ is represented by an ideal which can be factorised into finite number of prime ideals one can easily find $S$ such that $Cl_{S}(K)=0$. \\
%For $L/K$ finite Galois extension of number fields and $C_{L,S}$ as defined earlier.
In order to represent $C_{F,S} \subset C_{K,S}$ by the same set of places one must have $Cl_{S}(F)=0$ for subfield $F\subset L$. Suppose $L/M/K$ be a tower of number field extensions, then to represent $C_{K,S} \subset C_{L,S} $  and $C_{M,S} \subset C_{L,S}$ one must find $S$ so that $Cl_{S}(K)=Cl_{S}(M)=Cl_{S}(L)=0$.\\ 


\begin{remark}
Let $L/K$ be finite number fields extension, $G=\Gal(L/K)$ and $S$ satisfies::
\begin{enumerate} %[label=S\arabic*]
\item[S1.] for every $\mathfrak{p} \in S$, $\sigma(\mathfrak{p}) \in S$ where $\sigma \in \Gal(L/K)$ .\
\item[S2.] $S$ contains all ramified places of $L$.
\item[S3.] $S$ contains all the infinite places of $L$
\item[S4.] $S$ is large enough so that $Cl_{S}(F)=0$ for all $F\leq L$.

\end{enumerate}
Then one can represent $C_{F,S} \subset C_{L,S} $ for all subfields of $F\leq L$ and hence $H^{i}(G,C_{L,S}) \simeq H^{i}(G, C_L)$ because of $S4$. In fact, for every subgroup $G_F\leq G$ with  $F= L^{G_F}$ we get 
\[H^{i}(G_F,C_{F,S}) \simeq H^{i}(G_F, C_F).  \] 
\end{remark}
\begin{lemma}
Let $v$ be an infinite place of $L$, $i_v:L\hookrightarrow L_v$ be the corresponding embedding and  $G=\Gal(L/K)$. Consider the finitely generated  $G_v$-submodule $W$ of $L_{v}^{\times}$ which satisfies:
\begin{enumerate}
\item $i_v(U_{L,S}) \subset W$ and $W/i_v(U_{L,S})$ is torsion free and
\item $W\hookrightarrow L_{v}^{\times}$ induces an isomorphism in $G_v$-cohomology.
\end{enumerate}
Then there exists such a $W$ with extra property that if $W_v$ is any other $G_v$-submodule for which above conditions hold, there is a $G_v$-homomorphism $f: W \rightarrow W_v$ for which $f|_{i_v({U_{L,S}})}=\id$ and $f$ induces an isomorphism in cohomology.
\end{lemma}
\begin{proof}
Chinburg, Lemma 2.1, Debeerst Proposition 3.3.
\end{proof}
Suppose $v$ be the infinite place of $L$ over the place $u$ of $K$. Then $G_v=1$ if $v$ is real else $G_v =\{ 1, \sigma_v\}$ where $\sigma_v$ is complex conjugation automorphism. If $v$ is real then $K_u =\mathbb{R}=L_v$ and $H^{i}(G_v, L_{v}^{\times})=0$ for all $i$. In this case $W=i_v(U_{L,S})$ satisfies trivially the above lemma.
%If $v$ in complex infinte place then one can find $W$ through an algorithm presented in [Debeerst3.9].
\subsection{Finitely generated module for complex infinite place $v$}


%\subsection{complex module}
For every complex number field the big problem is to compute the finitely generated module. Although the algorithm is available in Debeerst but its very complicated to apply it. Once it is done one can apply the same procedure and can solve the global fundamental class.
The big remark is still left to find the alternative way of computing the module so that we compute this for higher field extension too. This will be as one of the future work.

\begin{definition}
 A submodule $N$ of an $R$-module M is called a pure submodule if 
 \[ \alpha N = N\cap \alpha M \text{  for every  } \alpha \in R.  \]

 \end{definition}
Consequences of the above definition :
1) Any direct summand of a module is a pure submodule.\\
2) A submodule $N$ of torsion-free module M is pure iff $\forall m \in M \& \forall \alpha \in R, \alpha m \in N \implies m \in N$.  \\
3) If $M/N$ is torsion-free, then $N$ is pure. If $M$ is torsion-free and $N$ is a pure submodule of $M$, then $M/N$ is torsion free.
\begin{theorem}
Let $M$ be a free module with a finite basis over a PID $R$ and $N\subset M$ be a submodule then $N$ is free iff $N$ is a direct summand of $M$.  
\end{theorem}
\begin{proof}
74.4 curtis Reiner
\end{proof}
\begin{theorem}
Let $R$ be an integral domain and $U$ be a rectangular matrix with coefficients in $R$. Then
\[A\sim diag\{\gamma_1, \ldots, \gamma_r,0,\ldots, 0\}, \hspace{4mm} \gamma_i \in R, \gamma_i \neq 0, \]
where $\gamma_i\mid \gamma_{i+1}$ for $1\leq i \leq r-1$.
\end{theorem}
\begin{proof}
Curtis , Theorem $16.6$.
\end{proof}
\begin{remark}
We will explain in details of the remark presented in the thesis.
Consider the finitely generated torsion free module $M$ and a cyclic group $G=\{1,\sigma\}$ acting on $M$.
Let $s= 1+\sigma$ and suppose the kernel of the map $s$ is $k$, which is a $\mathbb{Z}[G]$-submodule of $M$. In fact $k$ is free submodule of $M$ since $k\simeq I_{G}M$. Moreover, $k$ is a $\mathbb{Z}$ direct summand of $M$. Thus there exists a $\mathbb{Z}$-submodule $C$ of $M$ such that  $M=k\oplus C$ as $\mathbb{Z}$-modules. 
{\color{blue}Note that $C$ need not be a $\mathbb{Z}[G]$-submodule of M.}
Define $\{x+(s)\}\cdot y =xy$ for $x \in \mathbb{Z}[G], y \in k$ then $k$ is a left $R\simeq \mathbb{Z}[G]/(s)$ module.
$k$ is torsion free $R$-module. 

The image $(\sigma -1)M $ is in the kernel of $\sigma +1$. We know the $H^{-1}(G,M)=0$ that is , ${}_{N_{G}}M/ I_G(M) =1$. So the rank of $k$ and $C$ are equal and let it be $n$.  Let the basis for $k$ be  $\{b_1 \ldots b_n\}$ then using Smith normal form one can compute the basis for $C$ such as
  \[k= \mathbb{Z}b_{1} + \ldots \mathbb{Z}b_{n}\]
  and 
  \[ C = \mathbb{Z}e_{1}b_{1} + \ldots \mathbb{Z}e_{n}b_{n}\]
where $e_{i} \in \mathbb{Z}$.

From the relation $(\sigma -1)k \subset (\sigma -1)C \subset k$ one can obtain $\mathbb{Z}2b_{i} \subseteq \mathbb{Z}e_{i}b_{i} \subseteq \mathbb{Z}b_{i}$. This shows $e_{i} \in \{ 1,2\}$.\\
Let $r \in \mathbb{N}$ such that $e_{i}=1$ for $1\le i\le r$ and $e_{i}= 2$ for $r+1\le i \le n$. From above relation, the quotient $Q=C/(\sigma -1)k \simeq (\mathbb{Z}/2\mathbb{Z})^{r}$ is generated by the image $b_1^{*},\ldots b_r^{*}$ of $b_1, \ldots b_{r}$.
Define the surjective homoorphism $\phi: C \rightarrow Q$ such that $c\mapsto (\sigma -1)c +(\sigma -1)k$. Let the $\mathbb{Z}$-basis of $C$ be $\{c_1^{'}, \ldots c_n^{'}\}$ then $ k\geq r$ since $phi$ maps $C$ onto $Q$. Let $A=(\bar{a}_{ij}) \in Mat(\mathbb{Z}/2\mathbb{Z})$ be the representation matrix of the map $\phi$ such that 
\[ \phi(c_i^{'}) = \sum_{j=1}^{r}\bar{a}_{ij}b_{j}^{*}.\]

Suppose $\{c_1, \ldots , c_n \}$ be another $\mathbb{Z}[G]$- basis of $C$ such that $c_i= \sum_{j=1}^{k}u_{ij}c^{'}_j$ where $u_{ij} \in \mathbb{Z}$ for all $1\leq i \leq n$. Then the matrix $U=(u_(ij)$ is unimodular over $\mathbb{Z}$.
The matrix $\bar{A}$ is replaced by $\bar{U}\bar{A}$ when we replace the basis $\{c_1^{'}, \ldots c_n^{'}\}$ with $\{c_1, \ldots , c_n\}$. From above Theorem $4.3$, one can conclude that there exists unimodular matrix $U$ over $\mathbb{Z}$ so that 
\[\bar{U}\bar{A}= \begin{pmatrix}
 \bar{l}_1   &  0  & \ldots & 0 \\
 0 & \bar{l}_2 & \ldots & 0\\
 . & . & \ldots  & 0\\
  0 & 0 & \ldots & \bar{l}_r\\
   0 & 0 & \ldots  & 0\\
  . & . & \ldots  & .\\
  0 & 0 & \ldots  & 0\\ 
 \end{pmatrix}\]
takes the diagonal form with the entries $l_{i} \in  \mathbb{Z}$ such that $\bar{l}_{i}$ are non zero elements in $ \mathbb{Z}/2\mathbb{Z}$ for $1\leq i \leq r$.
%Using the Smith normal form diagonalizing $A$ over $\mathbb{Z}/2\mathbb{Z} $ we get a unimodular matrix $\overline{V} \in Gl_{k}(\mathbb{Z}/2\mathbb{Z})$.
%USing the lift one can obtain $V \in Gl_{k}(\mathbb{Z}) $ (Note that this lift happens in only for the case  $p=2$).
%Define $c_i = \sum_{j=1}^{k}v_{ij}c_{j}^{'}$ for $i= 1, \ldots k$.
 For such choice of $U$, we have 
\[\phi(c_i) = \begin{cases} \bar{l}_{i}b_i^{*} &\text{if }  1\le i \le r \\
               0 &\text{if } r+1\le i \le k
               \end{cases}.\]
%for suitable $l_{i} \in \mathbb{Z} \setminus 2\mathbb{Z} .$
Find $\gamma_{i} \in k$ such that $(\sigma -1)c_{i} - l_i b_i= (\sigma -1)\gamma_{i}$ for  $1\le i \le r$ (In fact $(\sigma -1)c_{i} - l_i b_i$ is square. for every $b_{i}^{*} \exists w \in C : phi(w)=b_{i}^{*}$ ).
and $(\sigma -1)c_j = (\sigma -1)\gamma_j$ for $r+1\le j \le n$.
Then clearly, $y_i= c_i - \gamma_i$ satisfies $\sigma y_i= l_i b_i + y_i$ for $1\le i \le r$ and $\sigma y_j= y_j$ for $r+1\le j \le n$.\\
For $1\le i \le r$, we have 
 $ \sigma y_i= l_i b_i + y_i$. Then\\ $  \sigma (\sigma y_i)= \sigma (l_{i} b_i + y_i)= (\sigma +1)(l_{i} b_i)+y_i=y_i$.
 Thus we obtain
\[M=(\mathbb{Z}b_1 \oplus \mathbb{Z}y_1) \oplus \cdots  \oplus(\mathbb{Z}b_r \oplus \mathbb{Z}y_r) \oplus \mathbb{Z}b_{r+1}\oplus \cdots  \oplus\mathbb{Z}b_{n} \oplus \mathbb{Z}y_{r+1} \oplus \cdots  \oplus\mathbb{Z}y_{k}\] 
 with $\mathbb{Z}[G]$-module isomorphism $\mathbb{Z}b_j \simeq \mathbb{Z}^{-}$ , $\mathbb{Z}y_j \simeq \mathbb{Z}^{+}$ for $j > r$ and 
 \[\mathbb{Z}[G] \simeq \mathbb{Z}b_i \oplus \mathbb{Z}y_i  \text{ , } 1\mapsto -y_i - (l_i^{'}+1)b_i\]
 where $l_{i}^{'}= (l_{i}-1)/2$.


\end{remark}

Once we have the  $ US/US.1= Z^{a}\oplus Z[G]^{b}$, we will apply the morphism $\psi$ so that the generators of $Z^{a-1}$ are $G_v$-invariant. For that we give the following algorithm to compute such generators:

\begin{algorithm}[H]
\caption{Generators:}
\label{}
\begin{algorithmic}[1]
%\begin{algorithm}
\REQUIRE  Generators of $Z^{a} $  $i.e. V=\{ x_1, \ldots x_a \}, \zeta$ as torsion unit and the $G_v$-action.
\ENSURE  Find $\{x'_1, \ldots x'_a\}$ such that $x'_2, \ldots x'_a $ are $Gv$-invariants.
\STATE $e_1 = x_1$.
\STATE For $i \in \{2..a\}$ do
\STATE find $a_1$ such that $\psi(e_1)= \zeta ^{a_1}\cdot e_1$,
\STATE find $a_i$ such that $\psi(x_i)= \zeta ^{a_i}\cdot x_i$,
\STATE find $e \& f $ such that $e\cdot a_1+f\cdot a_i= gcd(a_1,a_2)=d$.
\STATE $e'_1 = e\cdot e_1+ f \cdot x_i$.
\STATE $x'_i = -(a_2/d)\cdot e_1 + (a_1/d) \cdot x_2$.
\STATE $e_1 = e'_1$ and go to step 3.
\STATE \textbf{return} $\{e_1, x'_2, \ldots, x'_a\}$.

\end{algorithmic}
\end{algorithm}


Once we have the $Z[G_v]$-module we will induce the $Z[G]$-module. 
%\begin{lemma}
%Let $H\leq G$ be a subgroup. If $T$ is a left (right) trnsversal for $H$ in $G$, then $RG$ is free right (left) $RH$-module with $T$ as a basis.
%\end{lemma}
%\begin{proof}
%In our case $R$ is $Z$. induced module in BooK Lemma1.1.\\
%\[RG = \bigoplus_{t\in T}(RH)\]
%such that $RG$ is a free $RH$-module with basis $T$. A simlar argument for other case.
%\end{proof}

Using the proof of Lemma $3.1$ presented [Chinburg] and following the algorithm 3.7 from [Debeerst] we give the details to construct finitely generated $G_v$-module for infinite place of $L$.\\
\begin{algorithm}[H]
\caption{$G_v$-module for infinite place:}
\label{}
\begin{algorithmic}[1]
%\begin{algorithm}
\REQUIRE  $L/K$ finite Galois extension of number fields, $v$ a complex infinite place of $L$ with decomposition group $G_v=\{1, \sigma\}$.
\ENSURE  Finitely generated $\mathbb{Z}[G_v]$-module $W_v$.
\STATE Compute the $S$-unit group $U=U_{L,S}$ using [Coh00,Alg7.4.6] and compute its free part $U_0=U/U_{\tor}$ where $U_tor$ is torsion subgroup of $U$.
\STATE Choose $\theta \in U_{\tor}/U_{\tor}^2$ and define $(\mathbb{Z;U_{tor}})=U_{tor}\oplus \mathbb{Z}$ with $G_v$-action $\overline{(0,1)}= (\theta,1)$.
\STATE Compute $U_0= \mathbb{Z}^a \oplus \mathbb{Z}[G_v]^b$ for $a,b \in \mathbb{Z}$ and corresponding basis $\{\bar{x_1}, \ldots, \bar{x_a}, \bar{y_1}, \ldots, \bar{y_b}\}$ using theorem (74.3) of [Curtis] which we have described in the above remark. 
\STATE Compute the lifts of $U_0$ as $ x_i$ and $y_i$ respectively and find $c \in \mathbb{Z}$  such that $\bar{x_1}, \ldots, \bar{x_c} \notin U^{G_v}$ and  $x_{c+1}, \ldots, x_{a} \in U^{G_v}$.
\STATE Apply the \textbf{Algorithm $3$} for $\{x_1, \ldots, x_c\}$,  so that $x_2, \ldots, x_c$ are $G_v$-invariant and $x_1=\theta_1\cdot \eta_1$ where $\theta_1 \in U_{tor}$.
\STATE For $2\leq i \leq a$ choose the signs so that $i_v(x_i)\in \mathbb{R}_{>0}$.
%\STATE Define an isomorphism $\psi :(\mathbb{Z};U_{tor})\oplus \mathbb{Z}^{a-1}\oplus \mathbb{Z}[G_v]^b\rightarrow U$ such that for $i:=2, \ldots a$, $u_i:=\psi(g_i)$ have positive mebedding. 
\STATE Compute algebraically independent elements $\gamma_i \in \mathbb{C}$ satisfying $\gamma_i\bar{\gamma_i}=x_i$ and $\prod_{i=2}^{a}\gamma_{i}^{a_i + b_{i}\sigma} \in U$ then $a_i = b_i$ for $i=2,\ldots, a$.
\STATE \textbf{Return} the module $W_v= (\mathbb{Z};U_{\tor})\oplus\bigoplus_{i=2}^{a} \mathbb{Z}[G_v]\gamma_i\oplus \mathbb{Z}[G_v]^b$.

\end{algorithmic}
\end{algorithm}
This algorithm is long but we have managed to compute it. The problem is in the step $7$ to find $\gamma_{i}$ and then compute the algebra $\mathbb{Z}[G_v]\gamma_{i}$. But we take the abstract generators instead of $\gamma_{i}$.\\


From previous section we have finitely generated module $L_{v}^{\times}$ which is cohomologically isomorphic to $L_{v}^{\times}$ for an y place $v$ of $L$. In fact, $L_{v}^{f}= L_{v}^{\times}/ \exp(\mathsf{L}_v)$. One can construct the finitely generated module $L_{v}^{f}=W_v\subset \mathbb{C}^{\times}$ for infinite place $v$ of $L$. Then one can construct a finitely generated approximation to the $S$-id\`ele class group of $L$ by fixing a set of $G$-representatives $S(G)$ in $S$ and corresponding modules $L_{v}^{\times}$. We define
\[ I_{L,S}^{f}:= \bigoplus_{v\in S(G)} \ind_{G_v}^{G}L_{v}^{f}  \text{  and  } C_{L,S}^{f}:= I_{L,S}^{f}/U_{L,S}\] 
 which are finitely generated module.

\begin{proposition}
$H^{i}(G, I_{L,S}^{f}) \simeq H^{2}(G,I_{L,S})$ and $H^{i}(G, C_{L,S}^{f}) \simeq H^{2}(G,C_{L,S})$.
\end{proposition}
\begin{proof}
Chinburg prop2.1.
\end{proof}
Let $S_f$ and $S_{\infty}$ denote set of finite places and infinite places of $S$ respectively. For $v\in S_{\infty} $ the injection $ L_{v}^{f}=W_{v}\hookrightarrow L_{v}^{\times}$ induces an isomorphism in $G_v$-cohomology. Thus for every $v \in S$ we have $H^2(G_v, L_{v}^{\times}) \simeq H^2(G_v, L_{v}^{\times}) $ and therefore $I_{L,S}^{f}$ and $I_{L,S}$ are cohomologically isomorphic.
Let $S_{\infty,0}$ be a set of representatives for the $G$-orbits in $S_{\infty}$. Let $W_v$ be a submodule of $L_{v}^{\times}$ as of the Lemma $4.1$ for $v \in S_{\infty,0}$. Then define
\begin{eqnarray}
I_0  & = & \bigoplus \{  L_{v}^{f} : v \in S_f\} \\
I_{L,S}^{f} & = &I_0 \oplus \bigoplus\{  ind_{G_v}^{G}W_{v} : v \in S_{\infty,0}\}  \\
I_{L,S}^{q} & =  & I_0 \oplus \bigoplus \{  L_{v}^{\times} : v \in S_\infty\}
\end{eqnarray}
Identifying $\{  L_{v}^{\times} : v \in S_\infty\}$ with $\{  ind_{G_v}^{G}W_{v} : v \in S_{\infty,0}\}$ induces an injection $I_{L,S}^{f} \hookrightarrow I_{L,S}^{q}$ and then Lemma $4.1$ induces isomorphisms in cohomology. Since $U_{L,S} \subset W_v$ for $v\in S_{\infty,0}$, we have injections $U_{L,S}\rightarrow I_{L,S}^{f}$ and $U_{L,S}\rightarrow I_{L,S}^{q}$.
In this way we have $ C_{L,S}^{f}=I_{L,S}^{f}/U_{L,S}$ and $ C_{L,S}^{f}=I_{L,S}^{q}/U_{L,S}$.
Let us consider the exact diagram


\[\begin{tikzcd}
   0 \arrow{r} &  U_{L,S} \arrow{r}\ar[equal]{d}  & I _{L,S}^{f}  \arrow[hookrightarrow]{d}\arrow{r} & C _{L,S}^{f}\arrow[hookrightarrow]{d}\arrow{r} & 0 \\ 
   0 \arrow{r} & U_{L,S}\arrow{r} \ar[equal]{d} & I _{L,S}^{q}  \arrow{r} & C _{L,S}^{q}  \arrow{r} & 0\\\
    0 \arrow{r} &  U_{L,S} \arrow{r}  & I _{L,S} \arrow[two heads]{u}\arrow{r} & C _{L,S}\arrow[two heads]{u}\arrow{r} & 0. \\   
\end{tikzcd}\]
One can find using the consequence of five lemma that $C_{L,S}^{f}$ and $C_{L,S}$ are cohomologically isomorphic. Using this isomorphism we can compute the cohomology group of $C_{L,S}^{f}$.\\



Let $L/K$ be a finite Galois field extension with $G:=\Gal(L / K)$ and for any place $w$ of $L$ over place of $K$ with decomposition decomposition group $G_w:= \Gal(L_w / K_v)$. Then for every prime $v$ of $K$ we have the invariant map
\[\inv_{L_w/ K_v}: H^{2}(G_v, L_{w}^{\times}) \rightarrow \frac{1}{[L_w: K_v]} \mathbb{Z}/\mathbb{Z}.  \]
we obtain a canonical isomorphism
 \[ \inv_{L/K}H^{2}(G, I_L) \rightarrow \frac{1}{[L:K]}\mathbb{Z}/\mathbb{Z} , \hspace{2mm}\inv_{L/K}(c)=\sum_{w/v}^{} \inv_{L_w/K_v}(c_w)  \]
from the decomposition 
\[ H^{2}(G,I_{K}) \cong \bigoplus_{v}  H^{2}( G_v, L_{w}^{\times}).\]


\begin{algorithm}[H]
\caption{Global Fundamental Class:}
\label{}
\begin{algorithmic}[1]
\REQUIRE  A finite Galois extension $L/K$ of number fields with Galois group $G$.
\ENSURE  The global fundamental class $u_{L/K} \in H^{2}(G,C_{L,S}^{f})$. 
\STATE Find a cyclic extension $L'/K$ such that $[L':K]=[L:K]$ and choose a place $u_0$ of $K$ such that there is only on place $v'_{0}$ in $L'$ over $u_0$.
\STATE Compute the compositum $N=LL'$  with $\Gamma = \Gal(N/K)$ and $S$ a set of places satisfying $(S1)-(S3)$ for which $Cl_{S}(K)=Cl_{S}(L)=Cl_{S}(N)=0$. 
\STATE Compute $\mathsf{N}_{v} \subset \mathcal{O}_{N_v}$ for every $v\in S_{f}(\Gamma)$.
\STATE Compute $W_v$ using the algorithm 3.7 of Debeerst for every $v\in S_{\infty}$.
\STATE Compute $I_{N,S}^{f}, C_{N,S}^{f}$ and find the fixed modules $I_{L,S}^{f}=( I_{N,S}^{f})^{G}$, $C_{L,S}^{f}=( C_{N,S}^{f})^{G}$ where $G\subset \Gamma$ represents the Galois group of $L/K$.
\STATE Compute the cohomology group of $L/K$, $H^{2}(G,C_{L,S}^{f})$ and the boundaries $B^{2}(\Gamma, C_{N,S}^{f})$ using [Derekholt,Magma].
\STATE  For a fixed precision $k\in \mathbb{N}$, compute the local fundamental class of $L_{v'_{0}}/K_{u_0}$ using algorithm $2$ which represents the global fundamental class $u_{L'/K} \in Z^{2}(G', I_{L',S})$. Compute its inflation $\inf_{L'/K}^{N/K}(u_{L'/K}) \in C^{2}(\Gamma, C_{N,S}^{f})$.
\STATE \textbf{return} a generator $g$ of $H^{2}(G,C_{L,S}^{f})$ such that $\inf_{L'/K}^{N/K}(u_{L'/K}) - \inf_{L/K}^{N/K}(g) \in B^{2}(\Gamma, C_{N,S}^{f})$. 
\end{algorithmic}
\end{algorithm}
The computation of $S$-units is expensive. Because of this the modules $I_{N,S}$ and $C_{N,S}$ become large and consumes much time when the field is of high degree and $S$ contains many places. Since we can not reduce the degree of the field $N$ but we can minimise the number of places in $S$ satisfying all required conditions. We try to find the small primes in $S$ which are unramified and satisfies the condition $S4$. We can see this in the example presented below.
% The computation of completion in MAGMA is expensive once the pri is big

% To make the trivial       time Since the computation of $S$-Unit group is expensive if it contains many places and 

\begin{enumerate}
	\item Not needed to make all the class group to be trivial for all the subfields.
	\item we use our norm equation to compute the LFC.
	\item \textbf{Try to find the small primes which are unramified and which make class group to be trivial and this is possible}.
\end{enumerate}
\begin{ex}
\begin{lstlisting}

>K:=NumberField(CyclotomicPolynomial(23));
>S := [x[1]: x in Factorisation(Discriminant(MaximalOrder(K)))];  
>S;
[ 23 ]
>trivialSClassNumberPrimes(K:primes:=S);
[ 23, 47 ]
> S:=[23,2];
> trivialSClassNumberPrimes(K:primes:=S);
[ 2, 23 ]
>S:=[23,3];                                                     
>trivialSClassNumberPrimes(K:primes:=S);
[ 3, 23 ]
\end{lstlisting}
\end{ex}
The above example shows that the set of primes $[23,47]$ makes the class group trivial for every subfields of $K$. We also see that set  $[2,23]$ also makes $Cl_{S}(F)=0$ for all subfields $F$ of $K$. To compute the global fundamental class for number field $K$ we have to compute the local field $K_{v}$ for each $v\in S$ and have to work on it. Computation of $p$-adic field for big prime $p$ consumes much time and so on the functions applied on it. Therefore, it will be good to search for small prime numbers which satisfy our conditions. Also we know that the computation of local fundamental class is fast in unramified extension so we can also look for small unramified primes which make the class group trivial.\\

\begin{lstlisting}
Aslamali@CTM-Supports-MacBook-Pro ~ % magma
Magma V2.24-5     Mon Mar 16 2020 22:41:56 on CTM-Supports-MacBook-Pro
[Seed = 2946368287]
Type ? for help.  Type <Ctrl>-D to quit.
> x:=PolynomialRing(Integers()).1; 
> f := x^3 - 4*x + 1;                    
> AttachSpec("/Users/AslamAli/Desktop/Debeerst/magma/alispec");
> L:=SplittingField(f);
> g:=x^6-x^5-95*x^4+530*x^3-925*x^2+367*x+187;
> L1:=NumberField(g);
> time CohL,f1CL,gfc,rec := gfcCompositumcl(L,L1); 
Time: 255.100
> gfc;                                              
(1)
\end{lstlisting}
Let $N= L L1$ be the composite filed of $L$ and $L1$ of from example. We  optimised the algorithm and computed the suitable set of primes $S$ for $N$ so that it consumes only $255$ seconds to compute the global fundamental class for $L$. In this computation we take $S$ as a set of places of $N$ lying above the places $\{11,229,\infty\}$ of $\mathbb{Q}$ which satisfies the conditions required to implement the algorithm $5$. Let $US$ be the S-unit group of $N$. The place $\mathfrak{p}_1$ of $L1 $ over the place of $229$ of $\mathbb{Q}$ is undecomposed prime. We can compute the local fundamental class  $L1_{\mathfrak{p}_1} /\mathbb{Q}_{229}$ effectively and use it. Note that working on extension of $\mathbb{Q}_{229}$ is always expensive but in our case we rather choose undecomposed place over $229$ of $\mathbb{Q}$ because we target to minimize then number of places in $S$.  Since $N$ is totally real number field so we have $W_v=US$ the module for infinite place $v$ of $N$ and $G_v=\{id\}$. The $S$-units have $27$-generators so the induced module $\ind_{G_v}^{\Gamma}W_v$ is generated by $27*18=486$ elements $26*18=468$ free generators. Once we reduce the number of generators then we can optimise the time of  the computation global fundamental class. Debeerst takes $22$ minutes to compute it.


\subsection{gfc in relative number fields extension}

We are also able to compute the global fundamental class for relative number field extensions $L/K$. If we can find the exact place of $w$ of $L$ over $v$ of $K$ then one can  compute easily the completion $L_w$ over $K_v$. We present an algorithm to compute such completion in relative extension of number fields.
% In MAGMA we have the function "LocalField" which helps us to find the exact place $w$ of $L$ over $v$ of $K$.

\begin{Algorithm}
Input: $L/K$ be a finite Galois extension of number fields and Suppose  $f$ be the polynomial defining $L$ over $K$ and  $\mathbb{Q} < K \leq L.$\\
Output : Compute $L_{\mathfrak{P}} /K_{\mathfrak{p}}$ for $\mathfrak{P}$ over $\mathfrak{p}$ in $K$.
% such that $ L_{\mathfrak{P}}/K_p \leq M_{\mathfrak{p}} \leq L_{\mathfrak{P}}$.
\begin{enumerate}
\item Compute the completion $K_{\mathfrak{p}}$.
\item Convert the polynomial $f$ as $f_p$ over $K_{\mathfrak{p}}$.
\item Compute $K_{\mathfrak{p}}/f_p$. %Find the Local field extension using MAGMA.
\item  Using RamifiedRepresentaion Command one can find the $L_{\mathfrak{P}}$. 
\end{enumerate}
\end{Algorithm}
Since we have applied all our algorithm in MAGMA and it has the function "Local Field" which will compute the step $3$. Then using another function "RamifiedRepresentaion" we can convert in to $p$-adic field extension $L_\mathfrak{P}/K_{\mathfrak{p}}/\mathbb{Q}_p$.\\
\begin{lstlisting}
Magma V2.25-5     Thu Jun  4 2020 14:54:58 on nenepapa [Seed = 2921864985]
Type ? for help.  Type <Ctrl>-D to quit.
> x:=PolynomialRing(Integers()).1;
> K:=NumberField(x^2+5);
> y:=PolynomialRing(K).1;         
> L:=NumberField(y^2+1);
> Lp,mLp :=Completion(l,Factorisation(2*MaximalOrder(l))[1,1]);
> ChangePrecision(Lp,20);
Totally ramified extension defined by a map over Unramified extension defined by
a map over 2-adic field mod 2^10
> RLp, mRLp:= Lrelative_completion_check(l,L,K,2,20);
> RLp;
Unramified extension defined by the polynomial x^2 + x + 1
 over Totally ramified extension defined by a map over 2-adic field mod 2^10
Mapping from: FldNum: l to Unramified extension defined by the polynomial x^2 + 
    x + 1
 over Totally ramified extension defined by a map over pAdicField(2, 10) given 
by a rule


\end{lstlisting}
%> DefiningPolynomial(Lp,PrimeField(Lp));
%$.1^4 - (2 + O(2^10))*$.1^3 - (3 + O(2^10))*$.1^2 + (2^2 + O(2^10))*$.1 + 13 + 
 %   O(2^10)
%> DefiningPolynomial(RLp,PrimeField(RLp));  
%$.1^4 - (2 + O(2^10))*$.1^3 - (3 + O(2^10))*$.1^2 + (2^2 + O(2^10))*$.1 + 13 + 
  %  O(2^10)
\[\begin{tikzcd} [every arrow/.append style={dash}]
   L_p\arrow[dotted,dash]{rr}{\simeq} &  &  RL_p\\
 {} & L\arrow{lu} \arrow{ru} & K_p\arrow{u}\\
  \mathbb{Q}_{p}\arrow{uu} & K\arrow{ru}\arrow{u}  &  \mathbb{Q}_{p}\arrow{u}\\
  {} & \mathbb{Q}\arrow{lu}\arrow{ru}\arrow{u} & 
\end{tikzcd}\]
% Lp & L\arrow{l}{mLp} \arrow{r}{mLp'} & Lp'\\
 %{} & K\arrow{r}{mKp}\arrow{u}  & Kp\arrow{u}\\
% \mathbb{Q}_{p}\arrow{uu} & \mathbb{Q}\arrow{l}{m\mathbb{Q}_p}\arrow{r}{m\mathbb{Q}_p}\arrow{u} & \mathbb{Q}_{p}\arrow{u}
In the above example we see that $L_{p}$ is totally ramified over unramified extension but while computing the relative completion we have obtained unramified over ramified. But both completions $L_p$ and $RL_{p}$have the same defining polynomials over completion $\mathbb{Q}_{p}$.


Actually while computing the global fundamental class for relative number fields we have to work in many place for absolute fields. Because of this the computation time does not decrease for the small degree fields extension.\\

Using this algorithm one can set up for the computation of the global fundamental class for relative field extensions.


\begin{lstlisting}
Magma V2.23-11    Mon Apr 15 2019 12:05:43 on nenepapa [Seed = 3109138953] 
Type ? for help.  Type <Ctrl>-D to quit. 
> x:=PolynomialRing(Integers()).1; 
> K:=NumberField(PolynomialWithGaloisGroup(6,2)); 
> s:=Subfields(K,2)[1,1]; 
//Below its computed RayClassGroup(37^2*ZZ);
> l1:=NumberField(x^6+x^5-15*x^4-28*x^3+15*x^2+38*x-1); 
> IsCyclic(l1); 
true 
> L:=RelativeField(s,K);  
> IsSubfield(s,l1); 
true Mapping from: FldNum: s to FldNum: l1 
> L1:=RelativeField(s,l1); 
> Attach("ComplexModule.m"); 
> AttachSpec("spec"); 
>  time CohL,f1CL,gfc,comp,Req :=gfcCompositumcl(K,l1);                    
Time: 172.560 
> gfc; 
(1) 
> time CohL,f1CL,gfc,comp :=gfcCompositum_relative_check_update(L,L1); 
Time: 265.570 
> gfc; 
(1) 
> [x: x in CohomologyGroup(CohL,2)]; 
[ 
    (0), 
    (1), 
    (2) 
] 

\end{lstlisting}
To compute global fundamental class of relative number fields extension $L/K$, we have to find first the cyclic extension $L1$ over $K$ such that $[L:K]=[L1:K]$. Once $K=\mathbb{Q}$ then finding $L1$, working on the compositum $N$ of $L$ and $L1$ is an easier task than computing the case when when $K\neq \mathbb{Q}$.
\[\begin{tikzcd}[every arrow/.append style={dash}]
{}  & N\\
L\arrow{ur}\arrow{d} & L1 \arrow{u} \\
K\arrow{d}\arrow{ur}\\
\mathbb{Q}
\end{tikzcd}
\]

From above figure we see even if the  degree of number fields extension $L/K$ is very small, the computation time of $u_{L/K}\in H^{2}(\Gal(L/K), C_{L,S}^{f})$ depends up on the absolute degree of $L/K$ because in many places we have to use the absolute fields. To apply the algorithm of global fundamental class one needs to find suitable $S$ for which we must compute the class group of number fields. In MAGMA class group can be computed only for absolute field. If we can compute the class group of relative field then we can minimize some time of computations.

{\color{green}\textbf{Restricting}}
For $L/M/K$ be a tower of  Galois field extensions. Once we have the cohomology module $C$ for the extension $L/K$ then we  can restrict this module to for any subgroups of the Galois Group of $L/K$. In fact one can find the subgroup $H$ for the relative extensions $L/M$ and the we can restrict $C$ for this group $H$ by the command $Restriction(C, H)$ of $MAGMA$. And then we can compute the cohomology group and find the corresponding fundamental class for $L/M$ by restriction map because we know that  $u_{L/M} = res_{L/M}(u_{L/K} )$.\\
\begin{definition}
Cup Product::::::----->>>>>>>>
\end{definition}

Using the fundamental class $u_{L/K}$  we have the cup product which is an isomorphism 
\[ H^{r}(G(L/K), \mathbb{Z})\xrightarrow{\sim} H^{r+2}(G(L/K), C_{L}),   \]
for $r \in \mathbb{Z}$. Also for $L/K'/K$ be a tower of field extensions such that $L/K$ is Galois then the diagrams \\
\[\begin{tikzcd} [row sep=0.4em, column sep=1.5em]
H^{r}(G, \mathbb{Z})\arrow{dd}{\rest} \arrow{r}{\sim} & H^{r+2}(G, C_{L})\arrow{dd}{\rest}  &  H^{r}(G, \mathbb{Z})\arrow{r}{\sim} & H^{r+2}(G, C_{L}) \\
  &   \hspace{3.5cm}\text{and} &   & \\
H^{r}(G', \mathbb{Z})\arrow{r}{\sim} & H^{r+2}(G', C_{L})  & H^{r}(G', \mathbb{Z})\arrow{r}{\sim}\arrow{uu}{cor}  & H^{r+2}(G', C_{L})\arrow{uu}{cor}
%\arrow{from=A, to=B, Rightarrow}
\end{tikzcd}\]
are commutative where $G= \Gal(L/K)$ and $G'=\Gal(L/K')$.\\
Note: For $r=-2$, we obtain a canonical isomorphism 
\[ H^{-2}(G,\mathbb{Z})\cong G(L/K)^{ab}\rightarrow C_{K}/N_{L/K}C_{L},  \]
which is an inverse of the Artin map.





{\color{blue}

\subsection{gfc for complex field written above few} {\color{blue}Already written in Global Fundamental----}
Let U,mU be the S-unit group of a complex number field $K$. Let $U_0=U/U_t$ be the torsion free $Z[G]$  module where $G$ is the automorphism group of number field.
Since we have the order of decomposition group of infinite places of $K$ is cyclic of order 2. 
In this case we use the remark 3.6 of Debeerst or Chinburg paper to create a Module for such infinite place.
We have $Gv :={\sigma ,id}$ for infinite place $v$. 
$H^{-1}(Gv, U)=0$ so is  $H^{-1}(Gv, U_0)=0$. But $H^{-1}(Gv, U_t)\ne 0$ because $-1  \notin I_G(U_t)$.

Let $M=U_0, G:=G_v$. Then 
$N_G(M)=k= Ker(\sigma+1)= Z^n$. 
and $I_G(M)=(\sigma-1)M= Z^n$ because $H^{-1}(Gv, U_0)=0$ that is they are equal.\\
There exists another $Z[G]$ module $X$ such that $M=k+X$ reason ???\\
%(My opinion is we have the exact sequence $N_G \rightarrow Z[G]) \rightarrow Z\rightarrow 0$. \\
or one can also check that the intersection is trivial or their generators generate the module $M$.\\
$(\sigma -1)k\subset (\sigma -1)M \subset k.$
 we get the factor group as $Q=(\sigma -1)M  / (\sigma -1)k  \simeq (Z/2Z)^r $ an  images $b_i^{*}$ of $b_i$ generate $Q$.
 
 Define the surjective homomorphism $\phi : X \rightarrow Q$ such that $x\mapsto (\sigma-1)x+(\sigma-1)M$.
 Let $<a_i^{'}> $be basis of X. Then $k>r(see)$. Let $A:=(a_{ij})_{k\times r}$ be the representation matrix for $\phi$ such that 
 $\phi(a_{i}^{'})= \sum_{1}^{r} a_{i,j}b_{j}^{*}$.\\

By diagonalizing  $A$ over $Z/2Z$ by Echelon form we can find that$\overline{V} \in Gl_{k}(Z/2Z)$ and then corresponds to a lift $v \in Gl_{k}(Z)$ such that 
$a_i= \sum_{1}^{k} v_{i,j}a_j^{'}$  and $\phi(a_i)=c_{i}b_{i}^{*}$ for $1<i \leq e$ and $\phi(a_i)=0$ for $r+1\leq i \leq k$ where $c_i \in Z \setminus 2Z$.

}
